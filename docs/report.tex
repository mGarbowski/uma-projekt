\documentclass{article}
\usepackage{graphicx} % Required for inserting images
\usepackage{csvsimple} % Csv tables
\usepackage{adjustbox}
\usepackage[polish]{babel}
\usepackage[T1]{fontenc}
\usepackage[margin=0.5in]{geometry}

\usepackage{subcaption}

\title{UMA - Projekt}
\author{Mikołaj Garbowski, Michał Pałasz}
\date{Semestr 2024Z}

\begin{document}

\maketitle

\section{Introduction}

\section{Macierze pomyłek}
Przykładowe macierze pomyłek dla wszystkich zbiorów danych i klasyfikatorów.
Prezentowane macierze pomyłek są otrzymane z jednokrotnego wykonania predykcji przy podejściu z oddzielnym zbiorem uczącym i testowym (podzielone w proporcji $0.6$).

\begin{figure}[ht]
    \centering
    \begin{tabular}{ccc}
        \begin{subfigure}{0.3\textwidth}
            \centering
            \csvautotabular{results/confusion-matrices/ID3_Balance-scale.csv}
            \caption{ID3: Balance Scale}
        \end{subfigure} &
        \begin{subfigure}{0.3\textwidth}
            \centering
            \csvautotabular{results/confusion-matrices/ID3_Car.csv}
            \caption{ID3: Car}
        \end{subfigure} &
        \begin{subfigure}{0.3\textwidth}
            \centering
            \csvautotabular{results/confusion-matrices/ID3_NPHA.csv}
            \caption{ID3: NPHA}
        \end{subfigure} \\
        \begin{subfigure}{0.3\textwidth}
            \centering
            \csvautotabular{results/confusion-matrices/One-Vs-One_Balance-scale.csv}
            \caption{One vs One: Balance Scale}
        \end{subfigure} &
        \begin{subfigure}{0.3\textwidth}
            \centering
            \csvautotabular{results/confusion-matrices/One-Vs-One_Car.csv}
            \caption{One vs One: Car}
        \end{subfigure} &
        \begin{subfigure}{0.3\textwidth}
            \centering
            \csvautotabular{results/confusion-matrices/One-Vs-One_NPHA.csv}
            \caption{One vs One: NPHA}
        \end{subfigure} \\
        \begin{subfigure}{0.3\textwidth}
            \centering
            \csvautotabular{results/confusion-matrices/One-Vs-Rest_Balance-scale.csv}
            \caption{One vs Rest: Balance Scale}
        \end{subfigure} &
        \begin{subfigure}{0.3\textwidth}
            \centering
            \csvautotabular{results/confusion-matrices/One-Vs-Rest_Car.csv}
            \caption{One vs Rest: Car}
        \end{subfigure} &
        \begin{subfigure}{0.3\textwidth}
            \centering
            \csvautotabular{results/confusion-matrices/One-Vs-Rest_NPHA.csv}
            \caption{One vs Rest: NPHA}
        \end{subfigure}
    \end{tabular}
    \caption{Macierze pomyłek}
    \label{fig:confusion-matrices-grid}
\end{figure}

\newpage
\section{Metryki}
Do porównania jakości klasyfikacji modeli wykorzystujemy metryki:
dokładność, odzysk, precyzja, miara F, specyficzność, TP rate, FP rate.

Dla zastosowania standardowych metryk klasyfikacji binarnej do problemu klasyfikacji wieloklasowej stosujemy
podejście makro-uśredniania i mikro-uśredniania. Wyniki dla obu wariantów przedstawiono w tabelach.

Dla każdej metryki podajemy wartość średnią i odchylenie standardowe wyliczone przy
k-krotnej walidacji krzyżowej z $k=5$.


\subsection{Makro-uśrednianie}
\begin{table}[h!]
    \centering
    \begin{adjustbox}{valign=c, width=\textwidth}
        \csvautotabular{results/metrics/macro-1.csv}
    \end{adjustbox}
    \label{tab:metrics-macro-1}
\end{table}

\begin{table}[h!]
    \centering
    \begin{adjustbox}{valign=c, width=\textwidth}
        \csvautotabular{results/metrics/macro-2.csv}
    \end{adjustbox}
    \label{tab:metrics-macro-2}
\end{table}

\subsection{Mikro-uśrednianie}

\begin{table}[h!]
    \centering
    \begin{adjustbox}{valign=c, width=\textwidth}
        \csvautotabular{results/metrics/micro-1.csv}
    \end{adjustbox}
    \label{tab:metrics-micro-1}
\end{table}

\begin{table}[h!]
    \centering
    \begin{adjustbox}{valign=c, width=\textwidth}
        \csvautotabular{results/metrics/micro-2.csv}
    \end{adjustbox}
    \label{tab:metrics-micro-2}
\end{table}

\end{document}
